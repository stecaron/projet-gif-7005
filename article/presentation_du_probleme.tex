Lors de ce projet proposé par Coveo \href{https://www.coveo.com/fr}{\includegraphics[height=0.3cm]{coveo_logo}}, nous devions utiliser un historique de requêtes faites par des utilisateurs afin de développer un modèle de recommandation de document. 
Le but du modèle est de proposer une série de 5 documents d'intérêt en fonction de la recherche qui est faite par l'utilisateur et de certaines autres caractéristiques.


Toutefois, dans la plupart des approches les plus populaires, le modèle commence par extraire de l'information des documents cible et peut par la suite se définir une mesure de distance entre une requête et chacun des documents pour déterminer quel serait la meilleure correspondance requête-document. 
Malheureusement, pour ce projet, nous n'avons pas accès au contenu des documents que l'on souhaite prédire, mais bien à un jeu restreint de caractéristiques telles la source du document, son auteur et son titre.

Nous avions donc le choix entre attaquer cette problématique comme un problème de régression pour attribuer un score à chacun des documents pour recommander ceux dont le score est le plus élevé ou comme un problème de classification.
Puisque l'approche par régression demandait d'attribuer un score aux documents pour les données de test sur lesquelles nous basons notre modèle, nous avons décidé d'approcher la situation comme un problème de classification en utilisant les probabilités a posteriori de notre modèle pour définir les 5 documents les plus pertinents.