Dans le cadre de ce projet proposé par la compagnie Coveo \href{https://www.coveo.com/fr}{\includegraphics[height=0.3cm]{coveo_logo}}, nous devions utiliser un historique de requêtes faites par des utilisateurs afin de développer un modèle de recommandation de documents. 
L'objectif du modèle est de proposer une série de 5 documents d'intérêt en fonction de la recherche qui est faite par l'utilisateur et de certaines autres caractéristiques.
\break

Il existe plusieurs méthodes pour bâtir des systèmes de recommandation. Parmi ces méthodes, il en existe deux qui sont fréquemment utilisées en pratique. Ces méthodes sont le filtrage collaboratif et les systèmes basés sur le contenu. La première consiste à calculer des associtations entre des items (documents) ou des utilisateurs pour ainsi utiliser cette information pour faire une recommandation. Cette méthode à l'avantage d'être relativement simple à implémenter. La deuxième méthode consiste à utiliser les attributs des items ou des utilisateurs pour prédire les documents d'intérêt. Cette méthode a l'avantage d'apprendre des liens entre certains attributs pour rafiner la qualité des recommandations.

Dans ces différentes approches décrites, le modèle commence par extraire de l'information des documents cible et peut par la suite se définir une mesure de distance entre une requête et chacun des documents pour déterminer quelle serait la meilleure correspondance requête-document. Dans notre cas, nous n'avons pas accès au contenu détaillé des documents que l'on souhaite prédire, mais seulement à un jeu restreint de caractéristiques comme la source du document, son auteur et son titre.

Nous avions également le choix d'attaquer la problématique comme un problème de régression ou comme un problème de classification. Dans le premier cas, il faut attribuer un score à chacun des documents et ainsi recommander ceux dont le score est le plus élevé. Dans le deuxième cas, il faut tenter de prédire directement une classe, qui correspond à un document en particulier. 

Nous avons décidé d'utiliser un système basé sur le contenu dans un contexte de classification. Étant donné que nous avons beaucoup d'utilisateurs différents (environ 600) et beaucoup de documents différents (environ 6000), la méthode basée sur le filtrage collaboratif nous apparaissait moins efficace. De plus, certains utilisateurs sont inconnus par le système, ce qui rend plus complexe la tâche d'utiliser cette information. 
\break

Le reste du document est structuré de cette façon: la section 2 décrit notre approche de manière plus détaillée, la section 3 présente notre méthodologie expérimentale, la section 4 présente les résultats expérimentaux, la section 5 fait l'analyse de ces résultats. Finalement, la section 6 présente les différents constats et leçons que nous avons tirés de ce projet.