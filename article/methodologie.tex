
Avant même de créer les attributs, il était nécessaire de faire la jonction entre les différents jeux de données entres autres pour indiquer au modèle quelle requête a mené à quel document. 
D'abord, voici les différents jeux de données (avec le nombre d'observations) à notre disposition:

\begin{center}
  \begin{tabular}{ |c|c|c|c| } 
     \hline
     Type & E & V & T \\
     \hline
     \hline
     \textit{searches} & 52133 & 14895 & 7448 \\ 
     \hline
     \textit{clicks} & 24491 & 6920 & ?? \\ 
     \hline
  \end{tabular}
\end{center}

La colonne E correspond à l'ensemble d'entraînement, la colonne V à l'ensemble de validation et la colonne T correspond à l'ensemble de test. 
Pour ce dernier, nous avions accès aux recherches seulement et non au clicks. 
Comme on peut le voir, il y a plus de recherches que de clicks. 
Cela veut dire que certaines recherches n'ont mené à aucun click. 
Une même recherche peut également avoir mené à plusieurs clicks. 
En bref, il existe plusieurs façons de joindre ces recherches avec ces clicks. 
Puisqu'elles n'avaient pas de pouvoir prédictif, nous avons retiré les recherches qui n'ont pas mené vers un click. 
De plus, nous avons garder seulement le dernier click, en se basant sur le moment du click, pour une même recherche. 
Notre raisonnement derrière ce traitement est que le dernier click est probablement le "bon" et que les précédents ajoutent du bruit au modèle, car ils ne sont pas vraiment des documents d'intérêt.

\subsection{Prétaitements des données}

Maintenant que nous avons des observations avec des variables explicatives et des étiquettes, la prochaine étape consiste à effectuer certains prétraitements sur ce jeu de données. Nous avons fait plusieurs combinaisons de pré-traitements de données. Premièrement, nous avons sélectionné les variables que nous voulions inclure dans notre modèle prédictif, c'est à dire : \texttt{query\_expression}, \texttt{search\_nresults}, \texttt{user\_country} et \texttt{user\_language}.

Une imputation sur les données manquantes a aussi été effectuée. Lorsqu'une donnée était manquante, nous avons pris la valeur moyenne de cette variable pour en faire l'imputation.

Pour attaquer de façon claire le problème de l'optimisation des pré-traitements et de la sélection des hyperparamètres, nous avons utilisé le module \emph{pipeline} de la librairie Python \emph{sklearn}. Ce dernier nous permet de définir nos différentes étapes de pré-traitement par des classes dont les paramètres sont modifiables. En procédant ainsi, on peut faire une recherche en grille non seulement sur les hyperparamètres de nos modèles, mais aussi sur les différents prétraitements possibles.
\break

En ce qui concerne les transformations sur les variables, trois types de transformation ont été effectués.
Les trois trois transformations qui ont été faites sont une transformation des variables catégorielles en valeurs numériques, une normalisation des données textuelles et une vectorisation des données textuelles pour obtenir des résultats numériques utilisables par les modèles d'appentissage automatique.

Pour la transformation des variables catégorielles, nous avons simplement créé une variable indicatrice pour chaque modalité de la variable catégorielle.

Pour ce qui est de la normalisation des données textuelles, nous avons testé deux scénarios.
Dans le premier cas, nous gardons la donnée textuelle telle qu'elle en ne faisant aucune normalisation.
Dans le deuxième cas, on utilise l'objet \emph{PorterStemmer} de la librairie Python \emph{NLTK} pour faire du stemming sur nos requêtes.
Cette technique consiste à faire une série de traitements automtiques qui retirent les affixes des mots de la requête pour conserver leur racine. 
Cela permet de regrouper les différentes conjugaisons d'un mot.

Pour la vectorisation des textes de recherche, nous avons integré quatres types de vectorisation dans notre \textit{pipeline}.
Pour les deux premières méthodes, nous avons utilisé l'objet \textit{CountVectorizer} de la librairie Python \textit{sklearn}. 
Nous avons premièrement utilisé cet objet pour créer une matrice d'occurences de chacun des mots pour toutes nos requêtes.
Ensuite, avec le même objet, nous avons testé la création d'une matrice de présence de chacun des mots (semblable, mais en n'ayant que des résultats binaires).

Par la suite, nous avons testé la méthode \textit{tf-idf}. 
Cette méthode permet d'attribuer un poids à chaque mot de la requête en fonction de sa présence dans la requête, mais aussi de sa présence dans les autres requêtes. Un mot présent dans tous les documents aura donc un poids moins élevé qu'un mot présent seulement dans quelques requêtes, mais à une grande fréquence. On peut ainsi mieux quantifier le pouvoir discriminant des mots de la requête.

Finalement, nous avons aussi testé la vectorisation de nos requêtes selon \textit{Word2Vec} de la librairie Python \textit{gensim}.
Avec ce modèle, les mots sont projetés dans un espace vectoriel qui permet de capter les similarités entre des mots partageant un entourage semblable. En effet, l'entraînement du modèle de plongement de mots se fait en prédisant l'entourage d'un mot donné à partir de sa représentation vectorielle. On peut ainsi obtenir de bons résultats en généralisation puisqu'on peut plus facilement construire des classifieurs basés sur la représentation vectorielle et qui s'applique à plusieurs mots contrairement à l'utilisation d'une matrice de compte par exemple.

Nous avons aussi fait une version de notre code où l'on faisait du prétraitement sur les titres des documents. Nous avons fait du \textit{clustering} avec les titres de documents pour prédire des groupes de documents au lieu de prédire des documents uniques. Pour ce faire, nous avons utilisée retirer les mots d'arrêts des titres de documents. Par la suite, nous avons faite deux combinaisons de prétraitement. Nous avons utilisé l'objet \textit{PorterStemmer} et par la suite nous avons vectorisé les titres des dcouments à l'aide de la méthode \textit{tf-idf}. La deuxième combinaison était de seulement utiliser \textit{Word2Vec} avec le modèle pré-entrainé de google disponible sur le \href{https://code.google.com/archive/p/word2vec/}{Web}. Une fois que nous avions les attributs des documents, il était possible de faire du \textit{clustering} en utilisant l'objet \textit{KMeans} de la librairie Python \textit{sklearn}. 

\subsection{Modélisation}

Une fois que nous avons un jeu de données interprétable par un algorithme d'apprentissage supervisé, la prochaine étape consiste à tester différents types d'algorithmes et aussi différentes combinaisons d'hyperparamètres. Voici les différents algorithmes que nous avons testés:

\begin{itemize}
  \item Classifieur \textit{k}-PPV
  \item Perceptron multicouche
\end{itemize}

Pour la méthode le \textit{clustering} de document était utilisée, les modèles étaient entrainés à prédire des \textit{clusters} de documents. Il y avait donc beaucoup moins de classes à prédire. Lorsque le \textit{clustering} n'était pas utilisé, les modèles étaient entrainés à prédire directement le \texttt{document\_id} associé à la requête. Une fois que l'algorithme prédisait un \textit{cluster} de document, pour la validation nous utilisions les cinqs documents qui étaient les plus fréquents parmis le \textit{cluster} prédit.

\break
Pour le classifieur \textit{k}-PPV nous avons considéré deux hyperparamètres différents. En premier lieu, nous avons évidemment testé plusieurs valeurs pour le nombre de voisins (\textit{k}). Nous avons testé les valeurs suivantes: 1, 3, 9, 11, 15, 25 et 50. 
Nous avons également testé différentes fonctions de poids: \texttt{uniform} et \texttt{distance}.

Pour le perceptron multicouche, nous avons seulement testé la fonction d'activation \textit{ReLu}. Cependant, nous avons testé plusieurs topologies de réseaux: 1 couche cachée avec 100 neurones, 2 couches cachées avec 100 neurones chacune et 3 couches cachées avec également 100 neurones par couche.
\break

Pour faire l'entraînement, la validation et le test des différents types d'algorithmes mentionnés, l'approche classique consiste à séparer le jeu de données utilisées en trois partitions, chacune servant justement à faire l'entraînement, la validation et le test. Comme mentionné au début de cette section, les données mises à disposition par Coveo sont déjà regroupées selon cette approche. Il s'avère toutefois que le jeu de données de test ne comprend pas les cibles, soit les clicks, ce qui le rend inutilisable pour présenter une performance finale en test (ce jeu sera utilisé par Coveo pour faire leur propre calcul de la performance du modèle soumi). Pour pallier ce problème, le jeu de validation a donc été utilisé comme jeu pour rapporter la performance en test, et le jeu d'entraînement pour entraîner et valider les modèles testés. Étant donné une réduction des données utilisées pour entraîner l'ensemble des différents modèles testés, une validation croisée à 2 plis a été effectuée sur le jeu d'entraînement d'origine. Il est plutôt courant d'utiliser au moins 3 plis pour effectuer des validations croisées, mais ici, 2 plis ont été choisis afin de réduire le temps de calcul pour l'entraînement des modèles. Après avoir entraîné les modèles avec toutes les différentes configurations choisies, le jeu de données de validation soumi par Coveo a été utilisé pour rapporter la performance en test. Finalement, afin d'utiliser un maximum de données pour entraîner le modèle final, les deux jeux de données d'entraînement et de validation ont été exploités.
\break

Enfin, pour évaluer la performance de notre modèle et ainsi choisir la configuration optimale, il faut une mesure de performance. Coveo a défini ce qui constitue une bonne prédiction. Chaque document cliqué par un utilisateur après avoir effectué une recherche est considéré comme pertinent. Ce sont donc les cibles que le modèle doit prédire pour une recherche donnée. Une bonne prédiction doit prédire un ensemble d'au plus 5 documents parmi lesquels on doit retrouver au moins 1 document pertinent. Cette condition est formellement donnée par la fonction de perte suivante, qui indique qu'une perte de 1 résulte de l'absence de documents communs entre l'ensemble de 5 documents renvoyé par notre modèle de prédiction, et l'ensemble des documents cliqués pour la recherche en question :

\begin{gather}
l(y_i, \widehat{y_i}):=\left\{
                          \begin{array}{ll}
                            1 \quad\text{si } y_i \cap \widehat{y_i}=\emptyset; \\
                            0 \quad\text{autrement}.
                          \end{array}
                        \right.
\end{gather}


