En conclusion, si nous avions eu plus de temps ou de ressources, nous aurions aimé faire un entraînement plus complet des hyperparamètres de nos modèles, tester d'autres modèles prédictifs et incorporer des informations sur l'historique de recherche de l'utilisateur pour personnaliser les résultats retournés.
Nous aurions également été curieux de voir les gains de performance que nous aurions pu obtenir avec le contenu des documents puisque la plupart des approches classiques en recherche d'information s'appuient sur cela.
\break

Finalement, ce projet nous a permis de nous initier avec certains concepts du traitement de la langue naturelle, ce que nous avons survolé brièvement dans le cours. De plus, cela nous a permis de réaliser l'importance de bien comprendre et d'analyser les données que nous utilisons pour construire un modèle, car elles sont la base des choix de modélisation à faire. Nous avons également pu voir l'importance de décortiquer le problème en plusieurs petits problèmes afin de procéder incrémentalement sans perdre trop de temps à tester des caractéristiques inutiles. Ce projet a été une très bonne pratique pour voir un type de problématique réel en modélisation prédictive.
