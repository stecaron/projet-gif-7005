En conclusion, si nous avions eu plus de temps ou de ressources, nous aurions aimé faire un entraînement plus complet des hyperparamètres de nos modèles, tester d'autres modèles prédictifs et incorporer des informations sur l'historique de recherche de l'utilisateur pour personnaliser les résultats retournés.
Nous aurions également été curieux de voir les gains de performance que nous aurions pu obtenir avec le contenu des documents puisque la plupart des approches classiques en recherche d'information s'appuient sur cela.


Finalement, ce projet nous a permis de voir l'importance de bien comprendre et d'analyser les données que nous utilisons pour construire notre modèle, car elles guident beaucoup les choix de modélisation à faire. Également, nous avons pu voir l'importance de décortiquer le problème en plusieurs morceaux afin de procéder incrémentalement sans perdre trop de temps à tester des carctéristiques inutiles. Ce projet a été une très bonne pratique pour voir les problématiques réelles de la modélisation prédictive.
