Faire du gros blabla sale sur les résultats. Pourquoi notre score n'est pas si élevé que ça, comment on aurait pu améliorer l'efficacité des embeddings. Techniques qui fonctionnent le mieux et avantages/inconvénients des différentes techniques en production (temps d'entraînement, mémoire, etc.)

Sam: Selon moi, voici les points qui devraient être traités majoritairement pour l'analyse:
	- Discuter du pourquoi notre meilleur modèle est le meilleur (grand nombre de données permet l'entraînement d'un réseau de neurones complexe avec biais faible)
	- Pourquoi les embeddings n'apportent pas de performances accrues (Pas d'entraînement possible pour un réseau personnalisé à cause qu'on n'a pas assez d'entourage pour nos mots + modèle pré-entraîné de Google ne contient pas le même vocabulaire que nous, jase pas trop de Coveo).
	- Avantage du réseau de neurones comparativement au k-PPV en production, car on veut un temps d'évaluation très petit, surtout pour un engin de recherche.
	- Présenter pourquoi le clustering des documents n'améliore pas les résultats (sûrement difficulté à bien regrouper les documents avec seulement les titres qui fait en sorte que les caractéristiques des requêtes sous-jacentes ne sont pas si simliaires et difficiles à recouper pour créer des attributs)
\break

Début du vrai texte :
\break

Les résultats montrent que le meilleur score pour le modèle MLP est plus de deux fois supérieur qu'avec le meilleur score pour le modèle KNN (0.360 vs 0.166). Ceci montre que la problématique de recommandation de documents est probablement trop complexe pour être modélisée par un algorithme basé sur le voisinage comme le KNN. Les deux méthodes fonctionnent  bien avec un grand nombre de données et ont généralement un biais faible. Dans le cas du MLP, ce grand nombre de données a probablement permis l'entraînement d'un nombre important de neurones, ce qui est directement lié avec la capacité de notre classifieur à mieux modéliser le phénomène complexe de recommandation de documents. Par rapport à la problématique, le modèle KNN présente aussi un désavantage si on le compare au MLP en termes de temps d'évaluation lors de la présentation d'une nouvelle donnée. KNN doit stocker et utiliser les données pour déterminer le voisinage d'une nouvelle donnée en entrée. Cela peut représenter un obstable avec le nombre important de données utilisées pour bâtir le modèle, ce qui n'est pas souhaitable pour un engin de recherche qui doit idéalement donner une réponse rapidement pour satisfaire les exigence ergonomique d'un tel outil. En revanche, seul le modèle est conservé avecle MLP, pas les données, ce qui allège le calcul et donc le temps de réponse.
\break

	