Présentation des résultats avec tableaux, figures et tests statistiques.
On n'analyse rien ici, on ne fait que montrer ce que nous avons obtenue avec l'approche décrite plus haut.

Sam: Dans cette section, j'irais très simple et droit au but dans le style, pour telle étape, voici nos top 5 modèles et ça nous a pris Xh à rouler le tout.
Au total on a fait 9999 combinaisons de modèles/pré-traitements.
On laisse la section blabla pour l'analyse des résultats.

Commencer par présenter le baseline obtenu si on prédisait toujours les 5 documents les plus fréquents.

--- Vrai texte ici

Dans un premier temps, nous avons commencé par joindre les recherches et les clicks de la manière expliquée dans la section précédente. Voici le nombre d'observations que nous avons pour les ensembles d'entraînement et (E) de validation (V) après ces traitements:

\begin{center}
  \begin{tabular}{ |c|c|c|c| } 
     \hline
     Type & E & V \\
     \hline
     \hline
     \textit{searches/clicks} & 18571 & 6920 \\ 
     \hline
  \end{tabular}
\end{center}

Afin d'avoir une idée plus claire des performances reliées à nos modèles, nous avons commencé par nous bâtir un modèle \textit{baseline}. Ce modèle consiste simplement à piger au hasard 5 documents, selon les probabilités à priori des docuements dans le jeu de données d'entraînement. Pour ce modele, nous avons obtenu les résultats suivant: XXXXX

La majorité de nos résultats proviennent de la partie modélisation. Au total, nous avons testé XXXX combinaisons de prétraitements et d'hyperparamètres différents. Voici les 3 combinaisons les plus intéressantes selon nous :

\begin{itemize}
  \item Modele 1
  \item Modele 2
  \item Modele 3
\end{itemize}

[Parlez du temps d'entrainement de ces combinaisons de modeles aussi peut-etre?]


